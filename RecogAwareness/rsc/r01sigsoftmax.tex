%%%%%%%%%%%%%%%%%%%%%%% file template.tex %%%%%%%%%%%%%%%%%%%%%%%%%
%
% This is a general template file for the LaTeX package SVJour3
% for Springer journals.          Springer Heidelberg 2010/09/16
%
% Copy it to a new file with a new name and use it as the basis
% for your article. Delete % signs as needed.
%
% This template includes a few options for different layouts and
% content for various journals. Please consult a previous issue of
% your journal as needed.
%
%%%%%%%%%%%%%%%%%%%%%%%%%%%%%%%%%%%%%%%%%%%%%%%%%%%%%%%%%%%%%%%%%%%
%
% First comes an example EPS file -- just ignore it and
% proceed on the \documentclass line
% your LaTeX will extract the file if required
\begin{filecontents*}{example.eps}
%!PS-Adobe-3.0 EPSF-3.0
%%BoundingBox: 19 19 221 221
%%CreationDate: Mon Sep 29 1997
%%Creator: programmed by hand (JK)
%%EndComments
gsave
newpath
  20 20 moveto
  20 220 lineto
  220 220 lineto
  220 20 lineto
closepath
2 setlinewidth
gsave
  .4 setgray fill
grestore
stroke
grestore
\end{filecontents*}
%
\RequirePackage{fix-cm}
%
%\documentclass{svjour3}                     % onecolumn (standard format)
%\documentclass[smallcondensed]{svjour3}     % onecolumn (ditto)
\documentclass[smallextended]{svjour3}       % onecolumn (second format)
%\documentclass[twocolumn]{svjour3}          % twocolumn
%
\smartqed  % flush right qed marks, e.g. at end of proof
%
\usepackage{graphicx}
%
% \usepackage{mathptmx}      % use Times fonts if available on your TeX system
%
% insert here the call for the packages your document requires
%\usepackage{latexsym}
% etc.
%
% please place your own definitions here and don't use \def but
% \newcommand{}{}
%
% Insert the name of "your journal" with
% \journalname{myjournal}
%
\begin{document}

\title{Sigsoftmax%\thanks{Grants or other notes
%about the article that should go on the front page should be
%placed here. General acknowledgments should be placed at the end of the article.}
}
\subtitle{Kanai et al's Sigsoftmax properties}

%\titlerunning{Short form of title}        % if too long for running head

\author{TK}

%\authorrunning{Short form of author list} % if too long for running head

\maketitle

\begin{abstract}
Sigsoftmax properties
\keywords{softmax \and softmax bottleneck \and sigsoftmax}
% \PACS{PACS code1 \and PACS code2 \and more}
% \subclass{MSC code1 \and MSC code2 \and more}
\end{abstract}

\section{Introduction}
\label{intro}

Kamai et al, NIPS 2018: ``Sigsoftmax: Reanalysis of the Softmax Bottleneck''

It is different than having a set of multiple binary outputs and then normalizing them with softmax, as follows.

Sigsoftmax:
\begin{equation}
y_i = \frac{\exp(a_i) \sigma(a_i)}{\sum_{j=1}^K \exp(a_j) \sigma(a_j)}
\label{eq: sigsoftmax}
\end{equation}

Binary with softmax:
\begin{equation}
y_i = \frac{\exp(\sigma(a_i))}{\sum_{j=1}^K \exp(\sigma(a_j)) }
\label{eq: binary with softmax}
\end{equation}

\section{Desired properties}
\label{sec:1}


``As the alternative function to softmax, a new output function $f(z)$ and its $g(z)$ should have all of the following properties ...''

\paragraph{Theorem 5.} Sigsoftmax has the following properties:

1. Nonlinearity of $\log(g(a))$: $\log(g(a)) = 2 a - \log(1 + \exp(a))$.

2. Numerically stable:
\begin{eqnarray}
\frac{\partial \log y_i}{\partial a_j}
= \left\{
\begin{array}{ll}
(1 - y_j) \cdot (2 - \sigma(a_j)) &\quad i = j, \\
-y_j \cdot (2 - \sigma(a_j)) &\quad i \neq j.
\end{array}
\right.
\end{eqnarray}

3. Non-negative: $g(a_i) = \exp(a_i) \sigma(a_i) \geq 0$.

4. Monotonically increasing: $a_1 \leq a_2 \Rightarrow \exp(a_1)\sigma(a_1) \leq \exp(a_2)\sigma(a_2)$.

\paragraph{What we have.}
(1.) Kamai et al's sigsoftmax: 

\begin{equation}
g(a) = \exp(a) \cdot \sigma(a) = \frac{\exp(a)}{1 + \exp(-a)} = \frac{\exp(2 a)}{\exp(a) + 1}
\end{equation}

(2.) Sigmoid:

\begin{equation}
\sigma(a) = \frac{1}{1 + \exp(-a)} = \frac{\exp(a)}{\exp(a)+1}
\end{equation}

(3.) Property 1: 

\begin{equation}
\log(g(a)) = 2 a - \log(1+\exp(a))
\label{eq: property 1}
\end{equation}

(4.) Normalization:

\begin{equation}
y_i = \frac{g(a_i)}{\sum_k g(a_k)}
\end{equation}

(5.) Analyse partial derivative of log output:

\begin{eqnarray}
\frac{\partial \log(y_i)}{\partial z_j}
&=& \frac{\partial \log(g(z_i))}{\partial z_j} - \frac{\partial \log(\sum_k g(z_k))}{\partial z_j}
\nonumber \\
&=& \frac{1}{g(z_i)} \frac{\partial g(z_i)}{\partial z_j} - \frac{1}{\sum_k g(z_k)} \frac{\partial g(z_j)}{\partial z_j}
\nonumber \\
&=& \left\{
\begin{array}{ll}
- \frac{1}{\sum_k g(z_k)} \frac{\partial g(z_j)}{\partial z_j}
 & \quad, j \neq i, \\
\left(\frac{1}{g(z_j)}  - \frac{1}{\sum_k g(z_k)}\right) \frac{\partial g(z_j)}{\partial z_j}
 & \quad, j = i.
\end{array}
\right. 
\nonumber \\
&=& \left\{
\begin{array}{ll}
- \frac{g(z_j)}{\sum_k g(z_k)} \frac{1}{g(z_j)} \frac{\partial g(z_j)}{\partial z_j}
 & \quad, j \neq i, \\
\left(\frac{g(z_j)}{g(z_j)}  - \frac{g(z_j)}{\sum_k g(z_k)}\right) \frac{1}{g(z_j)} \frac{\partial g(z_j)}{\partial z_j}
 & \quad, j = i.
\end{array}
\right. 
\nonumber \\
&=& \left\{
\begin{array}{ll}
-y_j \cdot \frac{\partial \log(g(z_j))}{\partial z_j}
 & \quad, j \neq i, \\
(1-y_j) \cdot \frac{\partial \log(g(z_j))}{\partial z_j}
 & \quad, j = i 
\end{array}
\right. 
\end{eqnarray} 

From property 1 (Equation~\ref{eq: property 1})
and $\frac{d \log(1 + \exp(a))}{d a} = \frac{\exp(a)}{1 + \exp(a)} = \sigma(a)$, we get:

\begin{equation}
\frac{\partial \log(y_i)}{\partial z_j}
= \left\{
\begin{array}{ll}
-y_j \cdot (2 - \sigma(a_j))
 & \quad, j \neq i, \\
(1-y_j) \cdot (2 - \sigma(a_j))
 & \quad, j = i 
\end{array}
\right. 
\end{equation}

\end{document}
% end of file template.tex

